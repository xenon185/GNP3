\section{Auswertung}
\subsection{Zu: Messung von Amplituden- und Phasengang der Filterschaltungen}
\subsubsection{Grenzfrequenzen Tiefpässe/Hochpässe}
\noindent Alle Messwerte der Grenzfrequenzen wurden zusammen mit den vorausberechneten Werten in einer Tabelle dargestellt und verglichen.

	\begin{table}[h]
		\centering
		\begin{tabular}{c|c|c|c|c|c}
			$ $                       & $Tiefpass$ & $Hochpass$   \\
			\hline
			$Butterworth_{errech.}$   & $1.573kHz$ & $1.611kHz$   \\
			$Butterworth_{gemes.}$    & $1.538kHz$ & $1.596kHz$   \\
			\hline
			$Tschebyscheff_{errech.}$ & $1.578kHz$ & $1.606kHz$   \\
			$Tschebyscheff_{gemes.} $ &	$1.557kHz$ & $1.592kHz$   \\
			\hline
			$Bessel_{errech.}$        & $1.585kHz$ & $1.582kHz$   \\
			$Bessel_{gemes.}$         & $1.551kHz$ & $1.610kHz$
		\end{tabular}
		\caption{Vergleich: gemessenen und vorausbestimmte Grenzfrequenzen der verschieden Tiefpässe/Hochpässe}
		\label{tab:grenzfrequnzen_hp_tp_vorausberechnung}
   	  \end{table}

\subsubsection{Mittenfrequenz/Sperrfrequenz des Bandpasses/der Bandsperre}
\noindent Die gemessenen Frequenzen wurden zusammen mit den vorausberechneten Werten in einer Tabelle dargestellt und anschließend verglichen.

		\begin{table}[h]
			\centering
			\begin{tabular}{c|c|c|c|c|c}
				$ $          & $Mittenfrequ._{errech.} / Mittenfreuqu._{gemes.}$ & $Sperrfrequ._{errech.} / Sperrfrequ._{gemes.}$  \\
				\hline
				$Bandpass$   & $1.556kHz / 1.591kHz$       & $/$         \\
				\hline
				$Bandsperre$ & $/$              & $1.568kHz / 1.592kHz$  \\   
			\end{tabular}
			\caption{Gegenüberstellung: gemessene und vorausbestimmte Mittenfrequenz und Sperrfrequenz des Bandpasses sowie der Bandsperre}
			\label{tab:grenzfrequnzen_bs_bp_vorausberechnung}
		\end{table}
		
\newpage

\subsubsection{Gemessenen Frequenzen bei einer Phasenverschiebung der Tiefpässe}
\noindent Die gemessenen Frequenzen wurden in einer Tabelle dargestellt und anschließend verglichen.

		\begin{table}[h]
			\centering
			\begin{tabular}{c|c|c|c|c|c}
				$ $             & $-60^\circ $ & $-120^\circ$  \\
				\hline		
				$Butterworth$   & $1.046kHz$   & $2.381kHz$    \\
				\hline
				$Tschebyscheff$ & $898.250kHz$ & $1.400kHz$    \\ 
				\hline
				$Bessel$        & $1.229kHz$   & $3.225kHz$    \\
			\end{tabular}
			\caption{Gemessene Frequenzen bei einer Phasenverschiebung von $-60^\circ$ und $-120^\circ$ }
			\label{tab:phasenverschiebung_hp_tp_vorausberechnung}
		\end{table}
		
\noindent Aus den bei einer bestimmten Phasenverschiebung gemessenen Frequenzen ist es möglich die Koeffizienten der drei Filterarten zu bestimmen. 
\noindent Für einen Tiefpass bestimmter Art der 2. Ordnung gilt:

\small{
\begin{alignat*}{2}
arg\underline{H}_{TP}(\Omega) &= arg(V_0) - arctan(\frac{a_1 \cdot \Omega}{1-b_1\cdot \Omega^2}) \\
tan(arg\underline{H}_{TP}(\Omega)) &= tan(1) - \frac{a_1 \cdot \Omega}{1-b_1\cdot \Omega^2}\\
a_1 &= \frac{\Bigl(tan(1) - tan(arg\underline{H}_{TP}(\Omega))\Bigr)\Bigl(1-b_1 \cdot \Omega^2\Bigr)}{\Omega}\\
\frac{\Bigl(tan(1) - tan(arg\underline{H}_{TP}(\Omega_{-60^\circ}))\Bigr)\Bigl(1-b_1 \cdot \Omega_{-60^\circ}^2\Bigr)}{\Omega_{-60^\circ}} &= \frac{\Bigl(tan(1) - tan(arg\underline{H}_{TP}(\Omega_{-120^\circ}))\Bigr)\Bigl(1-b_1 \cdot \Omega_{-120^\circ}^2\Bigr)}{\Omega_{-120^\circ}}\\
\frac{\Bigl(tan(1) - tan(-60^\circ))\Bigr)\Bigl(1-b_1 \cdot \Omega_{-60^\circ}^2\Bigr)}{\Omega_{-60^\circ}} &= \frac{\Bigl(tan(1) - tan(-120^\circ)\Bigr)\Bigl(1-b_1 \cdot \Omega_{-120^\circ}^2\Bigr)}{\Omega_{-120^\circ}}\\
\frac{\Bigl( 3.289 \Bigr)\Bigl(1-b_1 \cdot \Omega_{-60^\circ}^2\Bigr)}{\Omega_{-60^\circ}} &= \frac{\Bigl(-0.175\Bigr)\Bigl(1-b_1 \cdot \Omega_{-120^\circ}^2\Bigr)}{\Omega_{-120^\circ}}\\
\rightarrow b_1 &= \frac{3.289\cdot \Omega_{-120^\circ}+0.175\cdot \Omega_{-60^\circ}}
{3.289\cdot \Omega_{-120^\circ}\cdot \Omega_{-60^\circ}^2 +0.175\cdot \Omega_{-60^\circ}\cdot \Omega_{-120^\circ}^2}
\end{alignat*}}

\newpage

\subsection{Zu: Sprungantworten der Tiefpässe}
\noindent Die Anstiegszeit, das Überschwingen sowie die Einschwingzeit der drei Tiefpässe wurden gemeinsam in einer Tabelle zusammengefasst und verglichen.

\begin{table}[h]
	\centering
	\begin{tabular}{c|c|c|c|c|c}
		$ $             & $Anstiegszeit$ & $Ueberschwingen$ & $Einschwingzeit$  \\
		\hline
		$Butterworth$   & $220\mu s$     & $4.86\%$        & $320\mu s$ \\
		\hline
		$Tschebyscheff$ & $220\mu s$     & $27.24\%$       & $1.06ms$   \\
		\hline
		$Bessel$        & $268\mu s$     & $0\%$           & $308\mu s$ \\
	\end{tabular}
	\caption{Anstiegszeit, Überschwingen und Einschwingzeit der drei Tiefpässe}
	\label{tab:sprungantworten_tp}
\end{table}

\noindent Die Werte wurden den Oszillogrammen entnommen. Diese sind im Anhang zu finden.
\newpage