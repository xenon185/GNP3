\section{Auswertung}
\subsection{Zu: Messung von Amplituden- und Phasengang}
\subsubsection{Grenzfrequenzen Tiefpässe/Hochpässe}
\noindent Alle Messwerte der Grenzfrequenzen werden zusammen mit den vorausberechneten Werten in einer Tabelle dargestellt und verglichen.
   	  
   	  \begin{table}[h]
   	  	\centering
   	  	\begin{tabular}{c|c|c|c|c|c|}
						   	&	& $f_{g,rech}$	& $f_{g,mess}$	& $\Delta f_g$	& $\Delta f_g$ [\%] \\
   	  		\hline
   	  		Butterworth		& TP& $1.573kHz$	& $1.538kHz$	& $35Hz$		& $2,23$\\
							& HP& $1.611kHz$	& $1.596kHz$	& $15Hz$		& $0,93$\\
			\hline
   	  		Tschebyscheff	& TP& $1.578kHz$	& $1.557kHz$	& $21Hz$		& $1,33$\\
				   	  		& HP& $1.606kHz$	& $1.592kHz$	& $14Hz$		& $0,87$\\
   	  		\hline
   	  		Bessel			& TP& $1.585kHz$	& $1.551kHz$	& $34Hz$		& $2,15$\\
				   	  		& HP& $1.582kHz$	& $1.610kHz$	& $28Hz$		& $1,77$
   	  	\end{tabular}
		\caption{Vergleich der Werte, Tiefpässe und Hochpässe}
		\label{tab:grenzfrequnzen_hp_tp_vorausberechnung}
   	  \end{table}
   	  
   	  \noindent Geringe Messabweichungen ergeben sich zunächst aus dem Messgraphen des Audioanalyzers, da der Cursor nicht immer auf genau -3dB eingestellt werden kann, die Messpunkte des Messgeräts variieren. Zudem haben die verwendeten Widerstände nur idealer Weise die nominalen Werte, diese variieren auch. Mit Abweichungen von maximal $2,23$\% kann von einer relativ guten Messung ausgegangen werden.
   	  

\subsubsection{Mittenfrequenz/Sperrfrequenz des Bandpasses/der Bandsperre}
\noindent Die gemessenen Frequenzen werden zusammen mit den vorausberechneten Werten in einer Tabelle dargestellt und anschließend verglichen. Im folgenden wird $f_m$ die Mittenfrequenz sein und $f_s$ die Sperrfrequenz.

		\begin{table}[h]
			\centering
			\resizebox{\textwidth}{!}{
			\begin{tabular}{c|c|c|c|c|c|c|c|c|}
							& $f_{m,rech}$	& $f_{m,mess}$	& $\Delta f_m$	& $\Delta f_m$ [\%]	& $f_{s,rech}$	& $f_{s,mess}$	& $\Delta f_s$	& $\Delta f_s$ [\%]\\
				\hline
				Bandpass	& $1.556kHz$	& $1.591kHz$	& $35Hz$		& $2,25$			& - 			& - 			& - 		& - \\
				\hline
				Bandsperre	& -				& -				& - 			& - 				& $1.568kHz$	& $1.592kHz$	& $24Hz$ & $1,53$\\   
			\end{tabular}
			}
			\caption{Vergleich der Werte, Bandpass und Bandsperre}
			\label{tab:grenzfrequnzen_bs_bp_vorausberechnung}
		\end{table}
		
		\noindent Auch hier ergeben sich die geringen Abweichungen durch die zuvor beschriebenen Umstände.
		
\newpage

\subsubsection{Gemessenen Frequenzen bei einer Phasenverschiebung}
\noindent Die gemessenen Frequenzen bei gewählten Phasenverschiebungen werden in einer Tabelle dargestellt und anschließend mit den errechneten Werten verglichen. Die folgenden Messungen beziehen sich auf die Tiefpässe der jeweiligen Arten.

		\begin{table}[h]
			\centering
			\begin{tabular}{c|c|c|c|c|}
								& $f_{mess,-60^\circ}$	& $f_{rech,-60^\circ}$	&$\Delta f_{-60^\circ}$& $\Delta f_{-60^\circ}$[\%]\\
				\hline		
				Butterworth		& $1.046kHz$			&						&	&\\
				\hline
				Tschebyscheff	& $898.250kHz$			&						&	&\\ 
				\hline
				Bessel			& $1.229kHz$			&						&	&\\
			\end{tabular}
			\caption{Gemessene Frequenzen bei einer Phasenverschiebung von $-60^\circ$}
			\label{tab:phasenverschiebung_tp_vorausberechnung_60}
		\end{table}
		
		\begin{table}[h]
			\centering
			\begin{tabular}{c|c|c|c|c|}
								& $f_{mess,-120^\circ}$	& $f_{rech,-120^\circ}$	&$\Delta f_{-120^\circ}$& $\Delta f_{-120^\circ}$[\%]\\
				\hline		
				Butterworth		& $2.381kHz$			&						&	&\\
				\hline
				Tschebyscheff	& $1.400kHz$			&						&	&\\ 
				\hline
				Bessel			& $3.225kHz$			&						&	&\\
			\end{tabular}
			\caption{Gemessene Frequenzen bei einer Phasenverschiebung von $-120^\circ$}
			\label{tab:phasenverschiebung_tp_vorausberechnung_120}
		\end{table}
		
\noindent Aus den bei einer bestimmten Phasenverschiebung gemessenen Frequenzen ist es möglich die Koeffizienten der drei Filterarten zu bestimmen. 
\noindent Für einen Tiefpass bestimmter Art der 2. Ordnung gilt:

\small{
\begin{alignat*}{2}
arg\underline{H}_{TP}(\Omega) &= arg(V_0) - arctan(\frac{a_1 \cdot \Omega}{1-b_1\cdot \Omega^2}) \\
tan(arg\underline{H}_{TP}(\Omega)) &= tan(0) - \frac{a_1 \cdot \Omega}{1-b_1\cdot \Omega^2}\\
a_1 &= \frac{\Bigl(- tan(arg\underline{H}_{TP}(\Omega))\Bigr)\Bigl(1-b_1 \cdot \Omega^2\Bigr)}{\Omega}\\
\frac{\Bigl(- tan(arg\underline{H}_{TP}(\Omega_{-60^\circ}))\Bigr)\Bigl(1-b_1 \cdot \Omega_{-60^\circ}^2\Bigr)}{\Omega_{-60^\circ}} &= \frac{\Bigl(- tan(arg\underline{H}_{TP}(\Omega_{-120^\circ}))\Bigr)\Bigl(1-b_1 \cdot \Omega_{-120^\circ}^2\Bigr)}{\Omega_{-120^\circ}}\\
\frac{\Bigl(- tan(-60^\circ))\Bigr)\Bigl(1-b_1 \cdot \Omega_{-60^\circ}^2\Bigr)}{\Omega_{-60^\circ}} &= \frac{\Bigl(- tan(-120^\circ)\Bigr)\Bigl(1-b_1 \cdot \Omega_{-120^\circ}^2\Bigr)}{\Omega_{-120^\circ}}\\
\frac{\Bigl( \sqrt{3} \Bigr)\Bigl(1-b_1 \cdot \Omega_{-60^\circ}^2\Bigr)}{\Omega_{-60^\circ}} &= \frac{\Bigl(-\sqrt{3}\Bigr)\Bigl(1-b_1 \cdot \Omega_{-120^\circ}^2\Bigr)}{\Omega_{-120^\circ}}\\
\rightarrow b_1 &= \frac{\sqrt{3}\cdot \Omega_{-120^\circ}+\sqrt{3}\cdot \Omega_{-60^\circ}}
{\sqrt{3}\cdot \Omega_{-120^\circ}\cdot \Omega_{-60^\circ}^2 +\sqrt{3}\cdot \Omega_{-60^\circ}\cdot \Omega_{-120^\circ}^2}
\end{alignat*}}

\newpage

\noindent Anschließend es ist möglich mit einer gegebenen Frequenz und $b_1$ den Koeffizienten $a_1$ zu berechnen:

\small{
	\begin{alignat*}{2}
	arg\underline{H}_{TP}(\Omega) &= arg(V_0) - arctan(\frac{a_1 \cdot \Omega}{1-b_1\cdot \Omega^2}) \\
	tan(arg\underline{H}_{TP}(\Omega)) &= tan(0) - \frac{a_1 \cdot \Omega}{1-b_1\cdot \Omega^2}\\
	\rightarrow a_1 &= \frac{\Bigl(- tan(arg\underline{H}_{TP}(\Omega))\Bigr)\Bigl(1-b_1 \cdot \Omega^2\Bigr)}{\Omega}\\
	\end{alignat*}}

\noindent Aus den Berechnungen ergeben sich folgende Ergebnisse: \\

	\begin{table}[h]
		\centering
		\begin{tabular}{c|c|c|c||c|c|c}
			$ $             & $ a_{1, errechnet} $ & $ a_{1, ideal} $ & $\Delta a_1$ [\%] 
							& $b_{1, errechnet} $ & $ b_{1, ideal} $ & $\Delta b_1$ [\%] \\
			\hline	
			$Butterworth$   & $1.4279$             & $1.414$          & $0.98$            
			                & $0.9498$             & $1$              & $5.02$  \\ 	  	
			\hline
			$Tschebyscheff$ & $1.076$              & $1.065$          & $1.03$ 
			                & $1.928$              & $1.931$          & $0.16$  \\  
			\hline
			$ Bessel$ 		& $1.353$	           & $1.362$          & $0.66$  
			                & $0.607$              & $0.618$		  & $1.78$
		\end{tabular}
		\caption{Gegenüberstellung: errechnete und ideale Koeffizienten }
		\label{tab:koeffizienten}
	\end{table}
	
\newpage

\subsection{Zu: Sprungantworten der Tiefpässe}
\noindent Die Anstiegszeit, das Überschwingen sowie die Einschwingzeit der drei Tiefpässe wurden gemeinsam in einer Tabelle zusammengefasst und verglichen.

\begin{table}[h]
	\centering
	\begin{tabular}{c|c|c|c|c|c}
						& Anstiegszeit 	& Überschwingen	& Einschwingzeit  \\
		\hline
		Butterworth		& $220\mu s$	& $4.86\%$		& $320\mu s$ \\
		\hline
		Tschebyscheff	& $220\mu s$	& $27.24\%$		& $1.06ms$   \\
		\hline
		Bessel			& $268\mu s$	& $0\%$			& $308\mu s$ \\
	\end{tabular}
	\caption{Anstiegszeit, Überschwingen und Einschwingzeit der drei Tiefpässe}
	\label{tab:sprungantworten_tp}
\end{table}

\noindent Die Werte wurden den Oszillogrammen entnommen. Diese sind im Anhang zu finden.

\noindent In Tabelle \ref{tab:sprungantworten_tp} fällt auf, dass bei dem Tschebyscheff Tiefpass ein relativ großer Überschwinger stattfindet. Werden die benutzten Widerstände überprüft, da diese die drei verschiedenen Schaltungen unterscheiden, kann erkannt werden, dass die Widerstände $R_a$, $R_b$, $R_c$ und $R_f$ bei jeder gleich bleiben. Der Widerstand $R_e$ ändert sich geringfügig und der Widerstand $R_d$ erfährt große Änderungen. Dadurch kann folgende Theorie aufgestellt werden: Wird der Widerstand $R_d$ erhöht, verringert sich das Überschwingen und wird dieser reduziert, erhöht sich das Überschwingen in der Sprungantwort. Diese Theorie kann mit einer Spice-Simulation bestätigt werden. Dadurch ist nun auch bekannt, dass sowie der Widerstand $R_e$ erhöht wird, das Überschwingen sich auch erhöht. Wird dieser reduziert, nimmt das Überschwingen auch ab.







