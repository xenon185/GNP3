\section{Auswertung}
\subsection{Zu: Messung von Amplituden- und Phasengang}
\subsubsection{Grenzfrequenzen Tiefpässe/Hochpässe}
\noindent Alle Messwerte der Grenzfrequenzen wurden zusammen mit den vorausberechneten Werten in einer Tabelle dargestellt und verglichen.
   	  
   	  \begin{table}[h]
   	  	\centering
   	  	\begin{tabular}{c|c|c|c|c|c|}
						   	&	& $f_{g,rech}$	& $f_{g,mess}$	& $\Delta f_g$	& $\Delta f_g$ [\%] \\
   	  		\hline
   	  		Butterworth		& TP& $1.573kHz$	& $1.538kHz$	& $35Hz$		& $2,23$\\
							& HP& $1.611kHz$	& $1.596kHz$	& $15Hz$		& $0,93$\\
			\hline
   	  		Tschebyscheff	& TP& $1.578kHz$	& $1.557kHz$	& $21Hz$		& $1,33$\\
				   	  		& HP& $1.606kHz$	& $1.592kHz$	& $14Hz$		& $0,87$\\
   	  		\hline
   	  		Bessel			& TP& $1.585kHz$	& $1.551kHz$	& $34Hz$		& $2,15$\\
				   	  		& HP& $1.582kHz$	& $1.610kHz$	& $28Hz$		& $1,77$
   	  	\end{tabular}
		\caption{Vergleich der Werte, Tiefpässe und Hochpässe}
		\label{tab:grenzfrequnzen_hp_tp_vorausberechnung}
   	  \end{table}
   	  
   	  \newpage

\subsubsection{Mittenfrequenz/Sperrfrequenz des Bandpasses/der Bandsperre}
\noindent Die gemessenen Frequenzen wurden zusammen mit den vorausberechneten Werten in einer Tabelle dargestellt und anschließend verglichen.

		\begin{table}[h]
			\centering
			\resizebox{\textwidth}{!}{
			\begin{tabular}{c|c|c|c|c}
							& Mittenfreq$_{rech}$	& Mittenfreq$_{mess}$	& Sperrfreq$_{rech}$& Sperrfreq$_{mess}$\\
				\hline
				Bandpass	& $1.556kHz$			& $1.591kHz$			& -					& -					\\
				\hline
				Bandsperre& -						& -						& $1.568kHz$		& $1.592kHz$		\\   
			\end{tabular}
			}
			\caption{Vergleich der Werte, Bandpass und Bandsperre}
			\label{tab:grenzfrequnzen_bs_bp_vorausberechnung}
		\end{table}
		
\newpage

\subsubsection{Gemessenen Frequenzen bei einer Phasenverschiebung}
\noindent Die gemessenen Frequenzen wurden in einer Tabelle dargestellt und anschließend verglichen.

		\begin{table}[h]
			\centering
			\begin{tabular}{c|c|c|c|c|c}
				$ $             & $-60^\circ $ & $-120^\circ$  \\
				\hline		
				$Butterworth$   & $1.046kHz$   & $2.381kHz$    \\
				\hline
				$Tschebyscheff$ & $898.250kHz$ & $1.400kHz$    \\ 
				\hline
				$Bessel$        & $1.229kHz$   & $3.225kHz$    \\
			\end{tabular}
			\caption{Gemessene Frequenzen bei einer Phasenverschiebung von $-60^\circ$ und $-120^\circ$ }
			\label{tab:phasenverschiebung_hp_tp_vorausberechnung}
		\end{table}
		
\noindent Aus den bei einer bestimmten Phasenverschiebung gemessenen Frequenzen ist es möglich die Koeffizienten der drei Filterarten zu bestimmen. 

 
\newpage

\subsection{Zu: Sprungantworten der Tiefpässe}
\noindent Die Anstiegszeit, das Überschwingen sowie die Einschwingzeit der drei Tiefpässe wurden gemeinsam in einer Tabelle zusammengefasst und verglichen.

\begin{table}[h]
	\centering
	\begin{tabular}{c|c|c|c|c|c}
						& Anstiegszeit 	& Überschwingen	& Einschwingzeit  \\
		\hline
		Butterworth		& $220\mu s$	& $4.86\%$		& $320\mu s$ \\
		\hline
		Tschebyscheff	& $220\mu s$	& $27.24\%$		& $1.06ms$   \\
		\hline
		Bessel			& $268\mu s$	& $0\%$			& $308\mu s$ \\
	\end{tabular}
	\caption{Anstiegszeit, Überschwingen und Einschwingzeit der drei Tiefpässe}
	\label{tab:sprungantworten_tp}
\end{table}

\noindent Die Werte wurden den Oszillogrammen entnommen. Diese sind im Anhang zu finden.
\newpage